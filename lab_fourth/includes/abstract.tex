% !TeX spellcheck = en_US
% !TeX encoding = UTF-8
\chapter*{Abstract}
In the digital age, the security of web applications is paramount due to the increasing reliance on online services for business, communication, and social interaction. This report discusses a comprehensive analysis of a deliberately vulnerable web application designed to demonstrate and understand common security flaws: SQL Injection (SQLi), Cross-Site Scripting (XSS), and Cross-Site Request Forgery (CSRF). These vulnerabilities are prevalent in many web applications and pose significant risks if left unaddressed. The purpose of this analysis is to provide a detailed exploration of the vulnerabilities, demonstrate exploitation techniques, and highlight the potential impacts on web application security.
The exploration of SQL Injection, Cross-Site Scripting, and Cross-Site Request Forgery vulnerabilities in this deliberately insecure web application underscores the importance of robust security practices in web development. By understanding these common vulnerabilities and their exploitation methods, developers and security professionals can better defend against such attacks. The analysis not only illustrates the devastating impact these vulnerabilities can have but also emphasizes the need for comprehensive input validation, proper session management, and the implementation of security mechanisms such as parameterized queries, output encoding, and anti-CSRF tokens. Ultimately, securing web applications requires a proactive approach to identifying and mitigating potential security flaws to protect sensitive data and maintain user trust.
