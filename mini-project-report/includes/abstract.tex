% !TeX spellcheck = en_US
% !TeX encoding = UTF-8
\chapter*{Abstract}
The Vulnerability Scanner project aims to enhance software security by developing a tool that identifies and reports security vulnerabilities within software projects. 
The tool provides functionalities for user authentication, project creation, and management, as well as the capability to scan code repositories or Python source files for various vulnerabilities. 
By leveraging advanced machine learning models, including LSTM and GPT API the system detects vulnerabilities and generates comprehensive reports. 
Users can rescan their code after implementing fixes to ensure improved security. 
The system architecture includes a user-friendly interface, a robust backend for service management, a secure database for storing credentials and project data, and pre-trained machine learning models for effective vulnerability detection. 
The workflow involves user login, project setup, code retrieval and conversion, vulnerability analysis, report generation, and rescanning. 
This approach ensures a continuous cycle of security assessment and improvement, addressing vulnerabilities such as XSS, Path Disclosure, Remote Code Execution, and Command Injection.