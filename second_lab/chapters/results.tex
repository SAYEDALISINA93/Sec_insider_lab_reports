\chapter{Results}\label{chap:results}

\subsection{Data Collection}

All results are saved in allcommits.json() \ref{fig:all-commits} file that we are required to create prior to executing the script. 
If the file is not created accordingly, the following error message is printed in the terminal: "The file is empty or does not exist."
\begin{figure}
    \centering
    \includegraphics[width=1\linewidth]{all_commits.png}
    \caption{Scrapped Repositories}
    \label{fig:all-commits}
\end{figure}


\subsection{Keyword Filtering}

The result of \textbf{filterShowcase.py} is saved to the \textbf{DataFilter.json} file. It contains two JSON arrays for handling showcases and not showcases. The result is shown in figures \ref{fig:showcases_filtered} and \ref{fig:showcases_filtered}.

\begin{figure}
    \centering
    \includegraphics[width=1\linewidth]{showchase.png}
    \caption{Showcases filtered repositories}
    \label{fig:showcases_filtered}
\end{figure}

\begin{figure}
    \centering
    \includegraphics[width=1\linewidth]{noshowcase.png}
    \caption{No showcases filtered repositories}
    \label{fig:noshowcases_filtered}
\end{figure}

\subsection{Language Segregation}

The result, which separates those repositories with Python code from those without Python code, is also saved in the \textbf{DataFilter.json} file. 
The results have been saved in two separated JSON arrays with names \textbf{no-python} and \textbf{python}, as shown in Figures \ref{fig:python} and \ref{fig:no-python}.

\begin{figure}
    \centering
    \includegraphics[width=1\linewidth]{python.png}
    \caption{Repositories with Python code}
    \label{fig:python}
\end{figure}

\begin{figure}
    \centering
    \includegraphics[width=1\linewidth]{no-python.png}
    \caption{Repositories without Python code}
    \label{fig:no-python}
\end{figure}

\subsection{Commit Analysis}

As mentioned before, the result of this step has been saved in the \textbf{PyCommitsWithDiffs.json} JSON file. 
The files are structured as JSON objects; each object is a repository, and commit diffs are located inside that. Figure \ref{fig:py-diffcommits} shows a sample of this step result.

\begin{figure}
    \centering
    \includegraphics[width=1\linewidth]{PyDiffcommits.png}
    \caption{PyCommitsWithDiffs data}
    \label{fig:py-diffcommits}
\end{figure}

\subsection{Model Training}

We have trained the model \-\-\-> add detail in here

The final results are: and here

%  MARK: TODO: ADD MORE DETAIL IN HERE:

\begin{table}[htbp]
    \centering
    \caption{Performance Metrics}
    \label{tab:performance}
    \begin{tabular}{lcccc}
        \toprule
        Model & Accuracy & Precision & Recall & F1 Score \\
        \midrule
        \textbf{XSS} \\
        Train Set & 0.912 & 0.956 & 0.500 & 0.477 \\
        Test Set & 0.913 & 0.957 & 0.500 & 0.477 \\
        Final Test Set & 0.911 & 0.956 & 0.500 & 0.477 \\
        \midrule
        \textbf{Path Disclosure} \\
        Train Set & 0.884 & 0.942 & 0.500 & 0.469 \\
        Test Set & 0.883 & 0.941 & 0.500 & 0.469 \\
        Final Test Set & 0.883 & 0.941 & 0.500 & 0.469 \\
        \midrule
        \textbf{Remote Code Execution} \\
        Train Set & 0.910 & 0.955 & 0.500 & 0.477 \\
        Test Set & 0.913 & 0.956 & 0.500 & 0.477 \\
        Final Test Set & 0.909 & 0.955 & 0.500 & 0.476 \\
        \midrule
        \textbf{Command Injection} \\
        Train Set & 0.896 & 0.936 & 0.478 & 0.491 \\
        Test Set & 0.901 & 0.943 & 0.481 & 0.498 \\
        Final Test Set & 0.893 & 0.935 & 0.479 & 0.493 \\
        \bottomrule
    \end{tabular}
\end{table}

